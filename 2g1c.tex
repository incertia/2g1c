\documentclass[letterpaper]{article}

\usepackage{incertia}
\usepackage{minted}

\pagestyle{fancy}
\lhead{Will Song}
\chead{2 GHCs 1 C - The Non-consensual Haskell (GHC) JIT}
\rhead{\today}

% no red boxes on parser error:
\makeatletter
\expandafter\def\csname PYGdefault@tok@err\endcsname{\def\PYGdefault@bc##1{{\strut ##1}}}
\makeatother
\renewcommand\fcolorbox[4][]{\strut#4}

\begin{document}

\section{Overview}

The Glorious Glasgow Haskell Compiler (GHC) is the world's most popular Haskell
compiler due to its speed, language extensions, and robustness. The Haskell
language itself boasts incredible type safety, purity, and lazy evaluation. All
of these things put together lead to a compiled native binary that looks
absolutely nothing like compiled C or C++ code, or anything else that compiles
into the standard C calling conventions. Instead, we have a binary that is
littered with \mintinline{asm}{jmp} instructions and several hundred symbols
whose names do not relate in any way to the source code that is almost
undecipherable. Here, we introduce 2G1C, a new tool that is designed to ease,
although it is still quite painful, the process of debugging compiled Haskell
without the source.

The goal of this project is to provide a framework in which users can redirect
the control flow GHC-compiled Haskell programs. For example, if we have the
following program

\begin{minted}{haskell}
main :: IO ()
main = print (2 + 3)
\end{minted}

we can inject it with a library containing the following code

\begin{minted}{haskell}
plus :: Integer -> Integer -> Integer
plus a b = unsafePerformIO (plus' a b)
  where plus' a b = do
          let c = a + b
          putStrLn (show a ++ " + " ++ show b ++ " = " ++ show c)
          return c
\end{minted}

transforming the standard output

\begin{minted}{text}
5
\end{minted}

to the following output

\begin{minted}{text}
2 + 3 = 5
5
\end{minted}

While the current state of the project does not allow for actual IO to be done
within replacement function due to unsolved technical issues, it does allow us
to modify the function definition. e.g. we are able to use

\begin{minted}{haskell}
plus :: Integer -> Integer -> Integer
plus a b = (toInteger . length) ([1..fromInteger a] ++ [1..fromInteger b])
\end{minted}

to change the behavior of \mintinline{haskell}{(+)} for negative values.

\section{GHC Basics}

GHC executes in several stages, best described by the following diagram

\begin{listing}[H]
\begin{minted}[fontsize=\small]{text}
                                                                          +---------+
                                                         LLVM backend /-->| LLVM IR |--\
                                                                      |   +---------+  | LLVM
                                                                      |                v
 +------------+ Desugar  +------+ STGify  +-----+ CodeGen  +-----+    |   NCG    +----------+
 | Parse tree |--------->| Core |-------->| STG |--------->| C-- |----+--------->| Assembly |
 +------------+          +------+         +-----+          +-----+    |          +----------+
                                                                      |                ^
                                                                      |      +---+     | GCC
                                                            C backend \----->| C |-----/
                                                                             +---+
\end{minted}
\end{listing}

We are particularly interested in the \texttt{STG} and \texttt{C--} intermediate
representations of the Haskell code, because they are closely related to GHC's
execution model and thus our ability to disturb GHC compiled code with foreign
code.

The \texttt{STG} machine has several registers, which are completely described
by \mintinline{C}{struct StgRegTable} within the GHC sources. Some of these
registers are mapped to hardware registers, and on the \texttt{x86\_64}
architecture, the mapping is as follows:

\begin{minted}{text}
BaseReg   -> r13
Sp        -> rbp
Hp        -> r12
R1        -> rbx
R2        -> r14
R3        -> rsi
R4        -> rdi
R5        -> r8
R6        -> r9
SpLim     -> r15
MachSp    -> rsp
F|D[1..6] -> xmm[1..6]
\end{minted}

where \mintinline{text}{BaseReg} is of type \mintinline{C}{struct StgRegTable
*}. This information is accessible via \mintinline{C}{void getregs(stg_regset_t
*)} when inside a C function hooked from Haskell.

The mapping between \texttt{STG} and \texttt{C--} is a relatively trivial yet
crucial mapping. In \texttt{STG}, expressions (variables) can consist of only a
single function application so something like

\begin{minted}{haskell}
num = 1 + 2 + 3 + 4
\end{minted}

will get transformed into

\begin{minted}{haskell}
num = let tmp = 1 + 2
          tmp' = tmp + 3
      in tmp' + 4
\end{minted}

and finally converted into the following \texttt{C--}

\begin{minted}[fontsize=\small]{C}
num_entry() //  [R1]
        { info_tbl: [(c1ad,
                      label: num_info
                      rep:HeapRep static { Thunk })]
          stack_info: arg_space: 8 updfr_space: Just 8
        }
    {offset
      c1ad: // global
          _rpV::P64 = R1;
          if ((Sp + 8) - 48 < SpLim) (likely: False) goto c1ae; else goto c1af;
      c1ae: // global
          R1 = _rpV::P64;
          call (stg_gc_enter_1)(R1) args: 8, res: 0, upd: 8;
      c1af: // global
          (_c1aa::I64) = call "ccall" arg hints:  [PtrHint,
                                                   PtrHint]  result hints:  [PtrHint] newCAF(BaseReg, _rpV::P64);
          if (_c1aa::I64 == 0) goto c1ac; else goto c1ab;
      c1ac: // global
          call (I64[_rpV::P64])() args: 8, res: 0, upd: 8;
      c1ab: // global
          I64[Sp - 16] = stg_bh_upd_frame_info;
          I64[Sp - 8] = _c1aa::I64;
          R2 = GHC.Num.$fNumInteger_closure;
          I64[Sp - 40] = stg_ap_pp_info;
          P64[Sp - 32] = sat_s19o_closure;
          P64[Sp - 24] = sat_s19p_closure+1;
          Sp = Sp - 40;
          call GHC.Num.+_info(R2) args: 48, res: 0, upd: 24;
    }
}
\end{minted}

with a bunch of extra information omitted. What is important to note is that
\texttt{C--} is written in continuation passing style (CPS), where results are
not consumed by the caller but rather consumed by a continuation passed to the
function. An example would be

\begin{minted}{haskell}
add_cps :: (Num a) => a -> a -> (a -> b) -> b
add_cps a b f = f $ a + b

print_three :: IO ()
print_three = add_cps 1 2 print
\end{minted}

This is further evidenced by the frequent occurrences of \mintinline{asm}{jmp
[rbp]} that occur in the disassembly of every GHC compiled program. This makes
debugging GHC compiled programs especially hard, because it is almost impossible
to examine the usual ``stack frame'' because all arguments are on the stack and
it is difficult to determine which entries are continuations and which entries
are arguments. Some work may be possible in C, because we can maybe check if the
potential info table pointed to by each entry is a good info table, but this
work has not been done yet (alternatively, check if it's a good closure).


What is good news is that we are able to redirect Haskell control flow back into
C by overwriting a conditional jump that occurs within the stack overflow check.
The actual implementation is covered in \mintinline{C}{void hook(void *src, void
*dst)}, but the overall goal is to rewrite

\begin{minted}{asm}
haskell_stub:
  lea rax, [rbp - $local_stack_space]
  cmp rax, r15
  jb .garbage_collection
  ; ...
.garbage_collection:
  ; ...
\end{minted}

into

\begin{minted}{asm}
haskell_stub:
  lea rax, [rbp - $local_stack_space]
  cmp rax, r15
  jmp hook_stub
  ; ...
.garbage_collection:
  ; ...
\end{minted}

where \mintinline{asm}{hook_stub} will save the GHC runtime state, call into a C
hook, restore the runtime state when the C hook returns, and continue executing
the Haskell code. More good news is that as long the \texttt{STG} arity and type
signature of two functions agree, we are able to retarget one stub into the
other at no additional cost, because the runtime will expect everything to be
laid out in the same way and everything else will sail smoothly. Unfortunately,
this does mean that we cannot overwrite typeclass-defined functions with
functions with a typeclass in the signature. In particular, the
\mintinline{haskell}{(+) :: Num a => a -> a -> a} from \mintinline{haskell}{Num}
has arity $1$ while \mintinline{haskell}{plus' :: Num a => a -> a -> a} has
arity $3$. We can jump through some hoops and define a new typeclass and replace
the typeclass ``virtual function table'' (\textit{vtable}) in a hook stub, but
that seems overly complicated. Instead, we overwrite entries in the typeclass
\textit{vtable} to point to our own Haskell code and that is sufficient.

Recall the assignment \mintinline{C}{R2 = GHC.Num.$fNumInteger_closure} from the
\texttt{C--} code. \mintinline{C}{GHC.Num.+_info} is a stub that will look up
the required \mintinline{haskell}{(+)} function and then return into
\mintinline{C}{stg_ap_pp_fast} (apply function to $2$ pointers) to apply the
correct function to the two arguments. The
\mintinline{C}{GHC.Num.$fNumInteger_closure} is actually an array that
corresponds to the definition of the \mintinline{haskell}{Num} typeclass,
containing closures to their respective functions. In particular, we have

\begin{minted}[fontsize=\small]{haskell}
-- | Basic numeric class.
class  Num a  where
    {-# MINIMAL (+), (*), abs, signum, fromInteger, (negate | (-)) #-}

    (+), (-), (*)       :: a -> a -> a
    negate              :: a -> a
    abs                 :: a -> a
    signum              :: a -> a
    fromInteger         :: Integer -> a

    {-# INLINE (-) #-}
    {-# INLINE negate #-}
    x - y               = x + negate y
    negate x            = 0 - x

-- | @since 2.01
instance  Num Integer  where
    (+) = plusInteger
    (-) = minusInteger
    (*) = timesInteger
    negate         = negateInteger
    fromInteger x  =  x

    abs = absInteger
    signum = signumInteger
\end{minted}

and

\begin{minted}[fontsize=\small]{asm}
(lldb) disas -s `*(uint64_t *)(0x1000e46f0)` -c 1
base_GHCziNum_CZCNum_con_info:
    0x10003c9a0 <+0>: inc    rbx
(lldb) disas -s `*(uint64_t *)(*(uint64_t *)(0x1000e46f0+0x08) & ~0x07)` -c 1
integerzmgmp_GHCziIntegerziType_plusInteger_info:
    0x10007c748 <+0>: lea    rax, [rbp - 0x18]
(lldb) disas -s `*(uint64_t *)(*(uint64_t *)(0x1000e46f0+0x10) & ~0x07)` -c 1
integerzmgmp_GHCziIntegerziType_minusInteger_info:
    0x10007d250 <+0>: lea    rax, [rbp - 0x18]
(lldb) disas -s `*(uint64_t *)(*(uint64_t *)(0x1000e46f0+0x18) & ~0x07)` -c 1
integerzmgmp_GHCziIntegerziType_timesInteger_info:
    0x10007bda0 <+0>: lea    rax, [rbp - 0x18]
(lldb) disas -s `*(uint64_t *)(*(uint64_t *)(0x1000e46f0+0x20) & ~0x07)` -c 1
integerzmgmp_GHCziIntegerziType_negateInteger_info:
    0x10007a348 <+0>: lea    rax, [rbp - 0x10]
(lldb) disas -s `*(uint64_t *)(*(uint64_t *)(0x1000e46f0+0x28) & ~0x07)` -c 1
integerzmgmp_GHCziIntegerziType_absInteger_info:
    0x10007a580 <+0>: lea    rax, [rbp - 0x8]
(lldb) disas -s `*(uint64_t *)(*(uint64_t *)(0x1000e46f0+0x30) & ~0x07)` -c 1
integerzmgmp_GHCziIntegerziType_signumInteger_info:
    0x10007b020 <+0>: lea    rax, [rbp - 0x8]
(lldb) disas -s `*(uint64_t *)(*(uint64_t *)(0x1000e46f0+0x38) & ~0x07)` -c 1
base_GHCziNum_zdfNumIntegerzuzdcfromInteger_info:
    0x10003c978 <+0>: mov    rbx, r14
(lldb) disas -s `*(uint64_t *)(*(uint64_t *)(0x1000e46f0+0x40) & ~0x07)` -c 1
error: error: ; SNIPPED
\end{minted}

Inside the demo project, this is accomplished with the assignment

\begin{minted}{C}
((void **)numinteger)[1] = dlsym(RTLD_SELF, "UnsafeZZp_unsafezuzzpzqzqzq_closure");
\end{minted}

\section{The main issue}

One of the biggest issues surrounding this entire project when we introduced
foreign Haskell code is the introduction of a separate GHC runtime. Since GHC
really likes statically linking everything, this means that when we inject an
application with foreign Haskell we will end up with two separate copies of the
runtime, all of which are completely unrelated to each other. Because we wish to
do interesting things with \mintinline{haskell}{unsafePerformIO :: IO a -> a},
which calls into the runtime, some work must be done to make sure that the two
runtimes behave well with each other.

We first initialize our local runtime by calling

\begin{minted}{C}
hs_init(&argc, &argv);
\end{minted}

with some bogus data, but this is not enough.
\mintinline{haskell}{unsafePerformIO} will return into the runtime, and more
specifically, it will return into the functions

\begin{enumerate}[1.]
\item
\mintinline{C}{maybePerformBlockedException}
\item
\mintinline{C}{stg_unmaskAsyncExceptionszh_ret_info}
\item
\mintinline{C}{tryWakeupThread}
\end{enumerate}

While some of these are avoidable, we especially do not want to go into
\mintinline{C}{tryWakeupThread} in our use case, it requires forcing our local
runtime to refer back to the original binary's runtime, at the expense of
rewriting some assembly instructions. In particular, we want to overwrite things
of the form

\begin{minted}{asm}
cmp    rcx, qword ptr [rip + 0x40af82]
  ; (void *)0x0000000101c707d8: stg_END_TSO_QUEUE_closure
\end{minted}

so that the \mintinline{C}{stg_END_TSO_QUEUE_closure} is the one in the original
binary. Luckily, all instances of this pattern refer to the same pointer, so we
can do this in one overwrite. The more complicated case occurs when we compare
via \mintinline{asm}{lea}.

\begin{minted}{asm}
lea    rdi, [rip + 0x45e0d1]     ; stg_END_TSO_QUEUE_closure
cmp    rbx, rdi
je     0x10181276c               ; <+140> at RaiseAsync.c:593
\end{minted}

So far, only one replacement is required within
\mintinline{C}{maybePerformBlockedException} to get non-IO replacements to run
without crashing, and trying actual IO has given me a headache but will likely
require more replacements elsewhere. Other times, when we try to do IO, we hit
the \mintinline{C}{default} case in \mintinline{C}{evacuate(StgClosure **)},
which is because \mintinline{C}{HEAP_ALLOCED_GC} is not returning the correct
value possibly because things are being allocated in the address spaces of two
different runtimes. Unfortunately, because
\mintinline{C}{USE_LARGE_ADDRESS_SPACE} is set by default in GHC compiled
programs, this is a macro and will require significant effort to fix, either by
messing with the allocator in the local runtime or by patching
\mintinline{C}{evacuate}.

\section{Results}

We are able to patch Haskell binaries to return potentially different values and
we are able to use \mintinline{haskell}{unsafePerformIO} as long as no actual IO
is being done. In particular, within the demo application of a postfix stack
calculator, we can replace \mintinline{haskell}{Integer} addition with

\begin{minted}{haskell}
unsafe_zp''' :: Integer -> Integer -> Integer
unsafe_zp''' a b = unsafePerformIO $ do
  c <- return . toInteger . length $ [1..fromInteger a] ++ [1..fromInteger b]
  return $ c + 1
\end{minted}

as a very wonky (and wrong) addition to produce the following output:

\begin{minted}[fontsize=\small]{text}
welcome to the postfix stack machine calculator
the PID for this process is 33534 for informational purposes
> 4 2 + 9000 +
9006
> -1 -2 +
-3
(injection happens here)
> 4 2 + 9000 +
9008
> -1 -2 +
1
>
\end{minted}

We also have a pretty printer for general closures and function closures, and
can work with them from inside C code as follows:

\begin{minted}[fontsize=\small]{text}
base GHC.Num +: STACK closure at 0x100226ef0...
base GHC.Num +: tag=0
base GHC.Num +: real addr=0x100226ef0
base GHC.Num +: info ptr=0x100130338
base GHC.Num +: info tbl=0x100130328
base GHC.Num +: closure type: FUN_STATIC (14)
base GHC.Num +: closure payload pointers: 0
base GHC.Num +: closure payload non-pointers: 0
base GHC.Num +: function info table at 0x100130310...
base GHC.Num +: slow_apply_offset=0xffc3ff48
base GHC.Num +: fun_type: ARG_P (5)
base GHC.Num +: arity=1
base GHC.Num +: fun_bitmap=P
plusInteger: STACK closure at 0x10022c212...
plusInteger: tag=2
plusInteger: real addr=0x10022c210
plusInteger: info ptr=0x1001949e0
plusInteger: info tbl=0x1001949d0
plusInteger: closure type: FUN_STATIC (14)
plusInteger: closure payload pointers: 0
plusInteger: closure payload non-pointers: 0
plusInteger: function info table at 0x1001949b8...
plusInteger: slow_apply_offset=0x10
plusInteger: fun_type: ARG_PP (15)
plusInteger: arity=2
plusInteger: fun_bitmap=PP
haskell_zp: STACK closure at 0x10246cce8...
haskell_zp: tag=0
haskell_zp: real addr=0x10246cce8
haskell_zp: info ptr=0x10200a460
haskell_zp: info tbl=0x10200a450
haskell_zp: closure type: FUN_STATIC (14)
haskell_zp: closure payload pointers: 0
haskell_zp: closure payload non-pointers: 0
haskell_zp: function info table at 0x10200a438...
haskell_zp: slow_apply_offset=0x18f065
haskell_zp: fun_type: ARG_PP (15)
haskell_zp: arity=2
haskell_zp: fun_bitmap=PP
found 'cmp rax, r15 -> jb ...'
hooked!
> 2 3 +
---(+)---
BaseReg=0x100231e18
Sp=0x42001e7e70
SpLim=0x42001e00c0
Hp=0x420007f010
HpLim=0x420007ffff
R1=0x420007a308
R2=0x4200070ee8
R3=0x100227f89
R4=0x10022cf99
R5=0x420007ec88
R6=0x0
MachSp=0x7ffeefbfafa8
R1: ??? closure at 0x420007a308...
R1: tag=0
R1: real addr=0x420007a308
R1: info ptr=0x100004568
R1: info tbl=0x100004558
R1: closure type: THUNK_1_0 (16)
R1: closure payload pointers: 1
R1: closure payload non-pointers: 0
R1: payload[0]=0x1002241e1
(SNIPPED)
*Sp: STACK closure at 0x42001e7e70...
*Sp: tag=0
*Sp: real addr=0x42001e7e70
*Sp: info ptr=0x1001de1d8
*Sp: info tbl=0x1001de1c8
*Sp: closure type: UPDATE_FRAME (33)
*Sp: closure layout: 0x1
*Sp+0x20: ??? closure at 0x420007ec10...
*Sp+0x20: tag=0
*Sp+0x20: real addr=0x420007ec10
*Sp+0x20: info ptr=0x1001de098
*Sp+0x20: info tbl=0x1001de088
*Sp+0x20: closure type: THUNK_2_0 (18)
*Sp+0x20: closure payload pointers: 2
*Sp+0x20: closure payload non-pointers: 0
*Sp+0x20: payload[0]=0x4200037930
*Sp+0x20: payload[1]=0x420007a308
6
>
\end{minted}

Despite my best efforts, I was unable to find anything remotely recognizable on
the stack or in any of the registers.

\end{document}
